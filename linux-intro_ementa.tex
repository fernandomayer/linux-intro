\documentclass[a4paper,12pt]{article}

\usepackage[brazilian]{babel}
\usepackage[T1]{fontenc}
\usepackage[utf8]{inputenc}
\usepackage[margin=2.5cm]{geometry}
\usepackage{hyperref}

\usepackage{titlesec}
\titleformat*{\section}{\large\bfseries\sffamily}
\titleformat*{\subsection}{\normalsize\bfseries\sffamily}
\titleformat*{\subsubsection}{\normalsize\bfseries\sffamily}
\titleformat*{\paragraph}{\normalsize\bfseries\sffamily}
\titleformat*{\subparagraph}{\normalsize\bfseries\sffamily}
\usepackage[protrusion=true,expansion=true]{microtype}
\usepackage[charter]{mathdesign} % serif: Bitstream Charter
\usepackage[scaled]{beramono} % truetype: Bistream Vera Sans Mono
\usepackage[scaled]{helvet}

\usepackage{xspace} % lida com os espaços depois dos comandos
\providecommand{\eg}{\textit{e.g.}\xspace}
\providecommand{\ie}{\textit{i.e.}\xspace}
\providecommand{\R}{\textsf{R}\xspace}

\title{Introdução ao uso do Linux}
\author{Fernando de Pol Mayer \and Luiz Ricardo Nakamura}
\date{Setembro, 2013}

\begin{document}
\maketitle
\thispagestyle{empty}

O Linux é um sistema operacional completo, criado em 1991 como
alternativa livre ao sistema Windows. Inicialmente utilizado por
programadores, e rotulado como ``difícil'' de ser operado por usuários
comuns, atualmente vem se tornando a plataforma de escolha de um número
crescente de usuários, incluindo o meio acadêmico, empresas e setores do
governo. A liberdade e a facilidade de utilização aliada a alta
performance promovida pelo Linux tem atraído cada vez mais usuários
interessados na utilização desse sistema, não apenas como uma
alternativa, mas como uma plataforma única de trabalho.

\paragraph{Público-alvo} Discentes e docentes do PPGEEA

\paragraph{Data e Horário} 02 e 03 de setembro (14:00--18:00 hs)

\paragraph{Local} Lab. A

\paragraph{Número de vagas} 20

\paragraph{Objetivo} Expor o sistema Linux, desmistificando sua
utilização. Interessados terão a oportunidade de realizar a instalação
em seus próprios computadores.

\paragraph{Inscrições} Email para \url{msegatto@usp.br}

\begin{center}
  \textbf{Ementa}
\end{center}

\paragraph{Módulo I --- Exposição} Introdução. Características do
Linux. Principais distribuições. Contas de usuários. Hierarquia de
arquivos. Comandos básicos. Exposição do sistema.

\paragraph{Módulo II --- Instalação} Partições e discos. Sistemas de
arquivos. Instalação do sistema. Instalação dos principais
\textit{softwares} (\R, \LaTeX, \ldots).

\end{document}
